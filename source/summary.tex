%----------------------------------------------------------------------------
\chapter*{Összefoglalás}\addcontentsline{toc}{chapter}{Összefoglalás}
%----------------------------------------------------------------------------

A szakdolgozatom keretében részletesen megismertem mind a JPorta rendszerét, mind a felé támasztott elvárásokat oktatói és hallgatói szempontból. Ezek alapján főként az aktuálisan legkritikusabbnak ítélt területek fejlesztésével foglalkoztam, azaz a dinamikus mezőkkel és az automatikusan kiértékelődő feladatok rendszerével.

Előbbin elért eredményeknek köszönhetően az oktatók már létrehozhatnak olyan dinamikus értékelési mezőket, melyek más dinamikus mezőket is használnak a számításaik során. Ezzel segítik az oktatókat a bonyolultabb mezők elkészítésében. Tipikusan ez a félév végi eredmény meghatározásánál lehet különösen hasznos.

Az automatikus kiértékelő rendszer fejlesztésének köszönhetően már lehetőség van oktatóknak és hallgatóknak is jogosultságot adni a feladatok létrehozására. Ez egy további lépést jelent a tárgyak széleskörű támogatására, hiszen így a feladatok elkészítésére nem csak a portál adminisztrátorait lehet megkérni, hanem minden arra jogosult személy kísérletezhet.

A kódlefedettség ellenőrző pedig lehetőséget teremt a C és C++ nyelvű feladatok esetén a tesztesetek értékelésére. Így könnyebb automatikusan kiszűrni a hibákat, működési zavarokat a beadott megoldásokban.

Végezetül pedig további hasznos funkciókra tettem javaslatot, melyek tovább növelhetik az oktatók és hallgatók elégedettségét a portállal szemben.