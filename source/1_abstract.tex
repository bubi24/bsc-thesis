%----------------------------------------------------------------------------
% Abstract in hungarian
%----------------------------------------------------------------------------
\chapter*{Kivonat}%\addcontentsline{toc}{chapter}{Kivonat}
A programozás korszerű oktatásában egyre inkább előtérbe kerül az önálló tanulás és gyakorlás, melynek célja a hallgatók motiválása a mélyebb tudás megszerzésére. Ezt a széleskörben elterjedt internet hozzáférés tette lehetővé, hiszen így bárhol és bármikor elérhetőek ezek a rendszerek mindenki számára.

Ennek támogatására jött létre a BME Irányítástechnika és Informatika Tanszékén először a CPorta, majd később annak utódja, a Jportaként ismert webes oktatást segítő rendszer. A portál feladata egyfelől az oktatók igényeit kielégítő adminisztrációs felület biztosítása, másfelől hogy lehetőséget teremtsen a hallgatók programozási feladatainak automatikus kiértékelésére és ellenőrzésére.

A minél nagyobb fokú megbízható automatizálás mind a hallgatóknak, mind az oktatóknak nagy segítséget biztosít. A hallgatók szinte azonnal értesülnek a feltöltött munkájuk esetleges hiányosságairól, hibáiról, mely megkönnyíti ezek javítását. Oktatói szempontból pedig nagy mértékben csökkenti a feladatok ellenőrzésére fordítandó munkát.

Szakdolgozatomban bemutatom a JPorta meglévő funkcionalitásait, különös tekintettel az adminsztrációs részekre és az automatikus kiértékelésre. Megtervezem és implementálom azon funkciókat, melyek akár oktatói, akár hallgatói oldalról növelik a portál értékét. Végezetül pedig javaslatot teszek további fejlesztési lehetőségekre.

\vfill

%----------------------------------------------------------------------------
% Abstract in english
%----------------------------------------------------------------------------
\begin{otherlanguage}{english}
\chapter*{Abstract}%\addcontentsline{toc}{chapter}{Abstract}

In the modern education of programming self-learning is increasingly emphasized to motivate students to acquire deeper knowledge. This is possible because of the widespread internet access, which allows these systems to be available from anywhere at any time. 

To offer proper self-learning environments the CPorta system was created at the Department of Control Engineering and Informatics at BUTE. Later CPorta got obsolete and difficult to improve so it was replaced by its successor knows as JPorta. The new portal was able to fulfill the requirements of teachers with its administration interface and also to provide an opportunity to automatically evaluate students' programming tasks.

This automation of programming tasks greatly helps both teachers and students. Students are able to see their resulst almost immediately for their uploaded solutions, which makes it easier to fix and try again. From a teacher's point of view it considerably reduces the time spent on chechking student submissions.

In my thesis I present the existing functionalities of JPorta, with particular regard to the administration and the automatic evaluation module. I design and implement new features that make administration even simpler for teachers. Finally, I propose further development opportunities.

\end{otherlanguage}
\vfill

