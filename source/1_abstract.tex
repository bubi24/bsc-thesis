%----------------------------------------------------------------------------
% Abstract in hungarian
%----------------------------------------------------------------------------
\chapter*{Kivonat}%\addcontentsline{toc}{chapter}{Kivonat}
A programozás korszerű oktatásában egyre inkább előtérbe kerül az önálló tanulás és gyakorlás, melynek célja a hallgatók motiválása a mélyebb tudás megszerzésére. Ezt a széleskörben elterjedt internet hozzáférés tette lehetővé, hiszen így bárhol és bármikor elérhetőek ezek a rendszerek mindenki számára.

Ennek támogatására jött létre a BME Irányítástechnika és Informatika Tanszékén először a Cporta, majd később annak utódja, a Jportaként ismert webes oktatást segítő rendszer. A portál feladata egyfelől az oktatók igényeit kielégítő adminisztrációs felület biztosítása, másfelől hogy lehetőséget teremtsen a hallgatók programozási feladatainak automatikus kiértékelésére és ellenőrzésére.

A minél nagyobb fokú megbízható automatizálás mind a hallgatóknak, mind az oktatóknak nagy segítséget biztosít. A hallgatók szinte azonnal értesülnek a feltöltött munkájuk esetleges hiányosságairól, hibáiról, mely megkönnyíti ezek javítását. Oktatói szempontból pedig nagy mértékben csökkenti a feladatok ellenőrzésére fordítandó munkát.

Szakdolgozatomban bemutatom a Jporta meglévő funkcionalitásait, különös tekintettel az adminsztrációs részekre és az automatikus kiértékelésre. Megtervezem és implementálom azon funkciókat, melyek akár oktatói, akár hallgatói oldalról növelik a portál értékét. Végezetül pedig javaslatot teszek további fejlesztési lehetőségekre.

\vfill

%----------------------------------------------------------------------------
% Abstract in english
%----------------------------------------------------------------------------
\begin{otherlanguage}{english}
\chapter*{Abstract}%\addcontentsline{toc}{chapter}{Abstract}


\end{otherlanguage}
\vfill

