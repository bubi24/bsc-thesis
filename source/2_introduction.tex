%----------------------------------------------------------------------------
\chapter*{Bevezető}\addcontentsline{toc}{chapter}{Bevezető}
%----------------------------------------------------------------------------

A programozói ismeretek elsajátításához merőben más módszertanra van szükség, mint egy irodalmi vagy jogász pályán. Az előbbi előnye, hogy megfelelő háttér biztosításával nagyban javítható a tanulási görbe, azonban sajnos az egyetemi körülmények között nincs lehetőség minden halltóval személyesen foglalkozni, hiszen ez óriási többlet munkát róna az oktatókra. Emiatt a hallgatók eredményes, mégis időtakarékos támogatása érdekében egyre nagyobb törekvés indult az automatizálása felé. Ezen megoldások számos előnnyel rendelkeznek:

\begin{itemize}
    \item Az értékeléshez nincs szükség a beadások letöltésére, azok saját számítógépen történő fordítására, futtatására, majd kiértékelésére.
    \item Azonnali visszajelzés a beadás sikerességéről a hallgatóknak.
    \item Egyéb beadáshoz kapcsolható metrikákkal is dolgohatunk: futási idő, memóriahasználat, stb.
    \item Automatikus pontszámítás a számított metrikák alapján.
    \item Határidők automatikus kezelése.
    \item Kommunikációs felület a hallagtó és oktató között.
\end{itemize}

\section*{Más automatikus kiértékelő rendszerek}\addcontentsline{toc}{section}{Más automatikus kiértékelő rendszerek}

\subsection*{Moodle}\addcontentsline{toc}{subsection}{Moodle}
A Moodle\cite{Moodle} egy teljes körű eLearning rendszer, mely nyílt forráskódú GNU GPL \cite{GNUGPL} licenc alatt készült. A Moodle-nek mára több, mint 120 ezer felhasználója van, a világ 232 országában \cite{MoodleStats}, ami a nyílt forráskód előnyeivel kombinálva óriási potenciált jelent. 

A Moodle általános lehetőséget biztosít az online oktatáshoz. Egyszerűen hozhatunk létre benne kurzusokat, melyekre a jelentkezést akár korlátozhatjuk is. A kurzusokon belül témakörökre bonthatjuk a tananyagot, melynek formája sokféle lehet: pdf fájlok, videók, külső weboldalak, stb. A résztvevők elsajátított tudásának ellenőrzése sem marad el: a Moodle általános rendszert biztosít online tesztek készítésére is, különféle jelleggel, mint pl. több válaszlehetőségből helyes(ek) kiválasztása, egy soros vagy éppen hosszabb saját szavas válaszok. A legtöbb fajta tesztnél lehetőségünk van a helyes válaszok megadására, így a portál azonnal ki is értékeli a beadást, ennek köszönhetően pedig a felhasználó azonnal értesül az elért eredményéről.

Elterjedtségének köszönhetően számos közösségi fejlesztésű modul is készült hozzá, melyek közül találhatunk szép számmal a programozás oktatására fókuszolóakat is. Ilyen például a Virtual Programming Lab \cite{VPL} \cite{VPLJournal}, mely támogat forráskódszerkesztést a böngészőben, programok futtatását, ellenőrzését és plágium ellenőrzést is. Mindezt a felhasználók a már megszokott Moodle környezetben érhetik el a kényelmes használat érdekében.

\subsection*{edX, Open edX}\addcontentsline{toc}{subsection}{edX, Open edx}
Az edX \cite{EDXAbout} egy nonprofit online kezdeményezés, melynek alapítói a Harvard University és a Massachusetts Institute of Technology. Ennek segítségével egyetemi szintű kurzusokat tartanak világszerte közel 10 millió felhasználóval és több, mint ezer kurzussal \cite{EDXReview}.

Az edX rendszer nem csak egyszerű tananyagok megtekintésére biztosít lehetőséget, de videókat és egyéb feladatokat is találunk a kínált kurzusokban. Emellett egyszerűen kapcsolatba léphetünk másokkal, akik a kurzusunkat hallgatják, ha éppen segítségre lenne szükségünk. A kurzusok többnyire ingyenesek, de sok esetben van lehetőségünk fizetés ellenében igazolást kapnunk a sikeresen elvégzett kurzusról.

A portálon lehetőségünk van programozással kapcsolatos kurzusok felvételére is. Ezek keretében pedig online kód írásra, fordításra, futtatásra és a helyesség ellenőrzésére is van mód, amik nélkül az önálló tanulás nagyon nehézkes lenne. Webes kódolást némely kurzusnál Codeboard\footnotemark integrációval van megoldva, melynek köszönhetően nem is kell a tanuláshoz a megfelelő környezeteket telepítenünk a saját gépünkre.
\footnotetext{A Codeboard (http://codeboard.io) egy böngésző alapú fejlesztő környezet a programozás oktatás segítésére. Támogatja az edX, moodle, cursera és egyéb platformokat.}

Az Open edX egy nyílt forráskódú platform, melyet az edX fejlesztett ki és tett szabadon hozzáférhetővé saját oktatási környezetek létrehozására. Közel a teljes szerver oldali kód Python nyelven íródott, melynek széleskörű ismerttségének (és a rendszer nyílt forráskódú voltának) köszönhetően bárki kiegészítheti a meglévő kódot.

\section*{A szakdolgozat felépítése}\addcontentsline{toc}{section}{A szakdolgozat felépítése}
TODO