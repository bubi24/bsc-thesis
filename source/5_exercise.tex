%----------------------------------------------------------------------------
\chapter{Automatizált feladatkiértékelő modul}\label{chapter:exercise}
%----------------------------------------------------------------------------

A JPorta egyik fő funkciója a beadott feladatmegoldások automatikus kiértékelése. Ennek tervezése során törekedtek az általános, könnyen bővíthető megoldás megtalálására, amikkel akár nem programozási jellegű feladatokat is ki lehet adni. \cite{DudiMsc}

Az elkészült modul blokkokat használ alapvető építő elemeinek, melyek egy-egy kiértékelési feladatot valósítanak meg. Ennek köszönhetően új igény esetén nem kell a meglévő blokkokat módosítani, csak az új funkciót megvalósító blokkot implementálni.

A blokkok közötti kommunikációt bemeneti és kimeneti csatlakozóik teszik lehetővé. Ezek szabadon összeköthetőek bármely más blokk ellenkező típusú csatlakozójával, pl. a felhasználói fájl blokk kimenetét ráköthetjük a GCC fordító blokk bemenetére, így lefordíhatjuk az adott forrásfájlt (blokkok típusait lásd lentebb). Ezen kívül lehetnek olyan paraméterei is egy blokknak, melyeket adminisztrátori felületen kell beállítani, pl. fordítás esetén a fordítónak átadott kapcsolók.

Ilyen meglévő blokkok a következők:

\begin{itemize}
    \item Specifikáció blokk: tartalmazza a feladat leírását, akár hallgató specifikus elemekkel (pl. neptun kód, név, egyedileg generált szám, stb.). 
    \item Szerzői fájl: a feladathoz oktatók által feltöltött fájlt tartalmazza. Szabályozható, hogy a felhasználók lássák-e ennek tartalmát. \label{authorfile}
    \item Felhasználói fájl blokk: hallgatói megoldásként feltöltött fájl.
    \item GCC fordító blokk: forrásfájlok fordítását végzi GCC fordítóval (\ref{fig:exercise_blokk}. ábra).
    \item Futtató blokk: futtatható fájlokat képes futtatni, kimeneteiket továbbítani.
    \item Szöveges ellenőrző blokk: bemenetén kapott szövegek összehasonlítását teszi lehetővé.
    \item Szkript ellenőrző blokk: különböző szkript nyelveken írt ellenőrzést tesz lehetővé, stb.
\end{itemize} 

\begin{figure}[p]
    \centering
    \resizebox{\textwidth}{!}{
        \includegraphics[]{exercise_blokk.png}
    }
    \caption{GCC fordító blokk adminisztrációs felülete}
    \label{fig:exercise_blokk}
\end{figure}

A beadás végső eredménye egy speciális blokk értéke lesz. Ennek a SubmissionResult blokknak egy bemente van, mely ha igaz a beadás sikeresnek tekinthető, ellenkező esetben sikertelen.

A blokkok kiértékelése (hasonlóan \aref{section:dynamic_dependencies}. pontban leírtakhoz) egy függőségi gráf felépítésével kezdődik, melynek kezdőpontja a fentebb említett SubmissionResult blokk. Csak azok a blokkok kerülnek kiértékelésre, amelyek (közvetve vagy közvetlen) függnek ettől a blokktól, hiszen a többi nem befolyásolja a beadás sikerességét.  Itt sem megengedettek a körkörös függőségek, tehát az elkészült gráfnak körmentesnek kell lennie.

\section{Jogosultságkezelés a JPortában}\label{subsection:permissions}

A JPorta által használt Django keretrendszer beépítetten tartalmaz egy egyszerű jogosultság kezelő rendszert. Ez lehetővé teszi különböző jogosultságok hozzárendelését felhasználókhoz, vagy felhasználói csoportokhoz. Minden Django modellhez \cite{DjangoModel} tartozik alapértelmezés szerint három jogosultság, melyeket a keretrendszer automatikusan hoz létre. Ezek a létrehozás (add), módosítás (change) és törlés (delete) jogosultságok. Sok esetben már ezek is elegek lehetnek számunkra, de bármikor létrehozhatunk új jogosultságokat, melyekkel személyre szabottabban tudjuk kezelni a hozzáférést egy-egy funkcióhoz. Ezeknek köszönhetően alapvteően egyszerűen kezelhetjük a jogosultságok kérdését \cite{DjangoAuth}. Fontos megjegyezni, hoyg Django-ban a \textit{superuser}-nek jelölt felhasználók automatikusan rendelkeznek minden jogosultsággal.

A JPorta ezen felül egyedi megoldást alkalmaz az egyes objektum példányok elérésének szabályozására, mely az access control list (ACL, \cite{ACL}) technikához hasonló. Az ACL minden objektumhoz rendel egy mátrixot, mely oszlopaiban a személyek (actor), soraiban pedig az egyes műveletek (action) találhatóak. Egy személy akkor hajthat végre egy műveletet, ha a mátrix általa és a művelet által meghatározott cellája igaz értéket tartalmaz.

\Aref{fig:acl}. ábra egy ilyen példát szemléltet: az objektumhoz tartozó hozzáférési mátrix értelmében Bob-nak csak olvasási, Alice-nak pedig olvasási és írási joga van. Emiatt Alice mind az olvasási, mind az írási műveletet sikeresen végrehajthatja, Bob azonban írási műveletet nem hajthat végre.

\begin{figure}[h]
    \centering
    \resizebox{\textwidth}{!}{
        \includegraphics[]{ACL.png}
    }
    \caption{ACL működése és egy hozzáférési mátrix}
    \label{fig:acl}
\end{figure}

A portál közvetlen jogosultságok helyett jogosultsági szinteket használ, tipikusan tulajdonosi (owner) és operátori (operator) szintekkel, de ez modellenként eltérő lehet. Ilyenkor ha egy felhasználót egy objektum tulajdonosának jelölünk, egyben operátori jogosultságokkal is felruházzuk. Ha egy felhasználónak nem kívánunk jogosultságot adni az adott objektumhoz, akkor nem jelöljük egyik szinten sem. Fontos megjegyezni, hogy a \textit{superuser}-nek jelölt felhasználók itt is automatikusan rendelkeznek minden jogosultsági szinttel.

Ez a megoldás kifinomultabbnak mondható az eredeti ACL technikához képest, mivel így különböző műveletek halmazát rendelhetjük egy adott szinthez, pl. az operátor jogosult megtekinteni és módosítani az adott objektum tulajdonságait, de a törléshez már tulajdonosi szintre van szükség.

\section{Jogosultságkezelés felülvizsgálata}

Az automatikus kiértékelő modulnál a jogosultság kezelés korábban nem készült el megfelelően, így ugyan hallgatói oldalról nézve minden jól működött, a feladatokat csak \textit{superuser} jogokkal rendelkező adminisztrátorok hozhatták létre. Ez pedig nagyban megnehezítette felsőbbéves hallgatók, laborvezetők bevonását a rendelkezésre álló feladatok bővítésére. Ennek oka, hogy ekkor minden más adatot is láttak volna a portálon, beleértve minden hallgató eredményeit, megoldásait, ami nem megengedhető.

Ezek értelmében a jogosultság kezelés kibővítésével a cél az volt, hogy egyszerűen és biztonságosan lehessen automatikusan kiértékelődő feladatok létrehozására és szerkesztésére. Az így elkészült rendszer alkalmazza mindkét fenti hozzáférés szabályozási módszert a maximális biztonság és testreszabottság elérésének érdekében.

Ennek elkészítéséhez először a Django jogosultságok ellenőrzését implementáltam, melyhez a beépített \textit{PermissionRequiredMixin} osztályt használtam. Ez egyszerűen teszi lehetővé a jogosultságok meglétének ellenőrzését, csak meg kell adnunk a szükséges jogosultság(ok) listáját, a többit pedig a keretrendszerre bízhatjuk. \cite{DjangoPermissionMixin}

Két jogosultságot használtam fel:

\begin{itemize}
    \item \textit{view\_exercise}: meglétével a felhasználó megtekintheti az összes eddig létrehozott automatikusan kiértékelődő feladatot (exercise). Ennek köszönhetően láthatja, ha már létezik hasonló feladat, mint amire szüksége van, illetve ötletet meríthet a többi feladatból. Emellett pedig a tanulási folyamatot is segíti, hiszen a már meglévő feladatok megértésével fény derülhet az addig rejtélyesnek tűnő működésre.
    \item \textit{create\_exercise}: meglétével a felhasználó létrehozhat új feladatokat. Ilyenkor lehetősége van új, üres feladatot létrehozni vagy egy korábbiról másolatot készíteni.
\end{itemize}

Ezek a jogosultságok fedik le általánosságban a hozzáférések szabályozását. A konkrét feladat példányok hozzáférésének konrtollálásához az előzőek kiegészítésére operátori és tulajdonosi szintet használtam. Ha egy felhasználóhoz egyik sincs hozzárendelve, akkor csak megtekintheti az adott feladatot, nem módosíthatja annak semmilyen elemét. A felhasználói szintek beállítása \aref{fig:exercise_perms}. ábrán látható.

\begin{itemize}
    \item Operátor (operator): az adott feladat példány alap adatait (cím, leírás és címkék), meglévő blokkjait, azok összeköttetéseit tudja módosítani. Lehetősége van más operátor szintű felhasználók felvételére is, illetve egy beadott feladatra újra lefuttatni a kiértékelési folyamatot.
    \item Tulajdonos (owner): az adott feladat példányhoz teljes jogkörrel rendelkezik. Az operátori szint jogain felül hozzáadhat, törölhet blokkokat, más személyeket tulajdonosi szintre emelhet és akár törölheti is a feladatot. Csak tulajdonosok publikálhatnak vagy vonhatnak vissza feladatokat\footnote{A feladat publikálása késznek jelöli azt, melyet csak ilyen állapotban lehet kurzushoz rendelni.}.
\end{itemize}

\begin{figure}[h]
    \centering
    \resizebox{\textwidth}{!}{
        \includegraphics[]{exercise_perms.png}
    }
    \caption{Feladat példányhoz tartozó jogosultsági szintek beállítása}
    \label{fig:exercise_perms}
\end{figure}

Végezetül a szerzői fájlok (ld. \aref{chapter:exercise}. fejezet) hozzáférését ellenőriztem. Itt van lehetőségünk a hallgatók elől elrejteni az adott fájlt, ugyanakkor ez csak a feladatbeadásnál való listázottságát módosítja csak. A fájlok webcím alapján közvetlenül továbbra is elérhetőek maradtak, így próbálgatással az összes titkosnak hitt szerzői fájl tartarlmát le lehetett kérdezni. Ennek megoldása egyszerűnek bizonyult, csak módosítani kellett a kérést kezelő függvényben a felhasználó és a visszaadandó fájl ellenőrzését. Eddig a fájl titkos voltától függetlenül visszaadásra került a tartalom.

\section{Kód lefedettség ellenőrzés}

\section{Feladatok (tesztek) importálása és exportálása}

\section{Feladatok csoportosítása}

\section{További lehetőségek}

\subsection{Plágiumkeresés}

\subsection{Verziókezelő támogatás}