\documentclass[12pt,a4paper,oneside]{report}             % Single-side
%\documentclass[11pt,a4paper,twoside,openright]{report}  % Duplex

%\PassOptionsToPackage{chapternumber=Huordinal}{magyar.ldf}

\usepackage{fontspec}
\setmainfont{Times New Roman}  % specify a font that supports required glyphs

\usepackage{amsmath}
\usepackage{amssymb}
\usepackage[english,magyar]{babel}
\usepackage{hyperref}
\usepackage{datetime}
\usepackage{enumerate}
\usepackage[thmmarks]{ntheorem}
\usepackage{graphics}
\usepackage{epsfig}
\usepackage{listings}
\usepackage{color}
%\usepackage{fancyhdr}
\usepackage{lastpage}
\usepackage{anysize}
\usepackage{sectsty}
\usepackage{setspace}  % Ettol a tablazatok, abrak, labjegyzetek maradnak 1-es sorkozzel!
\usepackage[hang]{caption}
\usepackage{pdfpages}
\usepackage{todonotes}
\usepackage{enumitem}
\usepackage{makecell}
\usepackage[all]{nowidow}
\usepackage{tabularx}

% Use “\cite{NEEDED}” to get Wikipedia-style “citation needed” in document
\usepackage{ifthen}
\let\oldcite=\cite
\renewcommand\cite[1]{\ifthenelse{\equal{#1}{NEEDED}}{\ensuremath{^\texttt{[citation~needed]}}}{\oldcite{#1}}}

\usepackage{wrapfig}
\usepackage{float}

%--------------------------------------------------------------------------------------
% Main variables
%--------------------------------------------------------------------------------------
\newcommand{\vikszerzo}{Nyíri Tamás}
\newcommand{\vikkonzulens}{Dr.~Szeberényi Imre}
\newcommand{\vikcim}{Feladatellenőrző rendszer továbbfejlesztése}
\newcommand{\viktanszek}{Irányítástechnika és Informatika Tanszék}
\newcommand{\vikdoktipus}{Szakdolgozat}
\graphicspath{{./figures/}}

%--------------------------------------------------------------------------------------
% Page layout setup
%--------------------------------------------------------------------------------------
% we need to redefine the pagestyle plain
% another possibility is to use the body of this command without \fancypagestyle
% and use \pagestyle{fancy} but in that case the special pages
% (like the ToC, the References, and the Chapter pages)remain in plane style

\pagestyle{plain}
%\setlength{\parindent}{0pt} % áttekinthetőbb, angol nyelvű dokumentumokban jellemző
%\setlength{\parskip}{8pt plus 3pt minus 3pt} % áttekinthetőbb, angol nyelvű dokumentumokban jellemző
\setlength{\parindent}{12pt} % magyar nyelvű dokumentumokban jellemző
\setlength{\parskip}{0pt}    % magyar nyelvű dokumentumokban jellemző

\marginsize{35mm}{25mm}{15mm}{15mm} % anysize package
\setcounter{secnumdepth}{0}
\sectionfont{\large\upshape\bfseries}
\setcounter{secnumdepth}{2}
\singlespacing
\frenchspacing

%--------------------------------------------------------------------------------------
%	Setup hyperref package
%--------------------------------------------------------------------------------------
\hypersetup{
    bookmarks=true,            % show bookmarks bar?
    unicode=true,              % non-Latin characters in Acrobat’s bookmarks
    pdftitle={\vikcim},        % title
    pdfauthor={\vikszerzo},    % author
    pdfsubject={\vikdoktipus}, % subject of the document
    pdfcreator={\vikszerzo},   % creator of the document
    pdfproducer={Producer},    % producer of the document
    pdfkeywords={keywords},    % list of keywords
    pdfnewwindow=true,         % links in new window
    colorlinks=true,           % false: boxed links; true: colored links
    linkcolor=black,           % color of internal links
    citecolor=black,           % color of links to bibliography
    filecolor=black,           % color of file links
    urlcolor=black             % color of external links
}

%--------------------------------------------------------------------------------------
% Set up listings
%--------------------------------------------------------------------------------------
\lstset{
	basicstyle=\scriptsize\ttfamily, % print whole listing small
	keywordstyle=\color{black}\bfseries\underbar, % underlined bold black keywords
	identifierstyle=, 					% nothing happens
	commentstyle=\color{white}, % white comments
	stringstyle=\scriptsize\sffamily, 			% typewriter type for strings
	showstringspaces=false,     % no special string spaces
	aboveskip=3pt,
	belowskip=3pt,
	columns=fixed,
	backgroundcolor=\color{lightgray},
} 		
\def\lstlistingname{lista}	

%--------------------------------------------------------------------------------------
%	Some new commands and declarations
%--------------------------------------------------------------------------------------
\newcommand{\code}[1]{{\upshape\ttfamily\scriptsize\indent #1}}

% define references
\newcommand{\figref}[1]{\ref{figure:#1}. ábra}
\renewcommand{\eqref}[1]{(\ref{eq:#1})}
\newcommand{\listref}[1]{\ref{listing:#1}.}
\newcommand{\sectref}[1]{\ref{sect:#1}}
\newcommand{\tabref}[1]{\ref{tab:#1}.}

\newcommand{\sectsep}{
	\par \nopagebreak
	\parbox{\linewidth}{
    	\centering\bigskip$\ast\quad\ast\quad\ast$\par\bigskip
	}
}

\DeclareMathOperator*{\argmax}{arg\,max}
%\DeclareMathOperator*[1]{\floor}{arg\,max}
\DeclareMathOperator{\sign}{sgn}
\DeclareMathOperator{\rot}{rot}
\definecolor{lightgray}{rgb}{0.95,0.95,0.95}

\author{\vikszerzo}
\title{\viktitle}
\includeonly{
	1_abstract,
	declaration,
	2_introduction,
	3_jporta,
	4_dynamic,
	5_exercise,
	acknowledgement,
	bibliography,
	summary,
	titlepage,
}
%--------------------------------------------------------------------------------------
%	Setup captions
%--------------------------------------------------------------------------------------
\captionsetup[figure]{
%labelsep=none,
%font={footnotesize,it},
%justification=justified,
width=.75\textwidth,
aboveskip=10pt}

\renewcommand{\captionlabelfont}{\small\bf}
\renewcommand{\captionfont}{\footnotesize\it}

%--------------------------------------------------------------------------------------
% Table of contents and the main text
%--------------------------------------------------------------------------------------
\begin{document}
\pagenumbering{arabic}
\spacing{1.5} %\onehalfspacing

\include{titlepage}
\includepdf[pages={1}]{feladatkiiras.pdf}
\include{declaration}
\tableofcontents\vfill
%----------------------------------------------------------------------------
% Abstract in hungarian
%----------------------------------------------------------------------------
\chapter*{Kivonat}%\addcontentsline{toc}{chapter}{Kivonat}
A programozás korszerű oktatásában egyre inkább előtérbe kerül az önálló tanulás és gyakorlás, melynek célja a hallgatók motiválása a mélyebb tudás megszerzésére. Ezt a széleskörben elterjedt internet hozzáférés tette lehetővé, hiszen így bárhol és bármikor elérhetőek ezek a rendszerek mindenki számára.

Ennek támogatására jött létre a BME Irányítástechnika és Informatika Tanszékén először a CPorta, majd később annak utódja, a Jportaként ismert webes oktatást segítő rendszer. A portál feladata egyfelől az oktatók igényeit kielégítő adminisztrációs felület biztosítása, másfelől hogy lehetőséget teremtsen a hallgatók programozási feladatainak automatikus kiértékelésére és ellenőrzésére.

A minél nagyobb fokú megbízható automatizálás mind a hallgatóknak, mind az oktatóknak nagy segítséget biztosít. A hallgatók szinte azonnal értesülnek a feltöltött munkájuk esetleges hiányosságairól, hibáiról, mely megkönnyíti ezek javítását. Oktatói szempontból pedig nagy mértékben csökkenti a feladatok ellenőrzésére fordítandó munkát.

Szakdolgozatomban bemutatom a JPorta meglévő funkcionalitásait, különös tekintettel az adminsztrációs részekre és az automatikus kiértékelésre. Megtervezem és implementálom azon funkciókat, melyek akár oktatói, akár hallgatói oldalról növelik a portál értékét. Végezetül pedig javaslatot teszek további fejlesztési lehetőségekre.

\vfill

%----------------------------------------------------------------------------
% Abstract in english
%----------------------------------------------------------------------------
\begin{otherlanguage}{english}
\chapter*{Abstract}%\addcontentsline{toc}{chapter}{Abstract}

In the modern education of programming self-learning is increasingly emphasized to motivate students to acquire deeper knowledge. This is possible because of the widespread internet access, which allows these systems to be available from anywhere at any time. 

To offer proper self-learning environments the CPorta system was created at the Department of Control Engineering and Informatics at BUTE. Later CPorta got obsolete and difficult to improve so it was replaced by its successor knows as JPorta. The new portal was able to fulfill the requirements of teachers with its administration interface and also to provide an opportunity to automatically evaluate students' programming tasks.

This automation of programming tasks greatly helps both teachers and students. Students are able to see their resulst almost immediately for their uploaded solutions, which makes it easier to fix and try again. From a teacher's point of view it considerably reduces the time spent on chechking student submissions.

In my thesis I present the existing functionalities of JPorta, with particular regard to the administration and the automatic evaluation module. I design and implement new features that make administration even simpler for teachers. Finally, I propose further development opportunities.

\end{otherlanguage}
\vfill


%----------------------------------------------------------------------------
\chapter*{Bevezető}\addcontentsline{toc}{chapter}{Bevezető}
%----------------------------------------------------------------------------

A programozói ismeretek elsajátításához merőben más módszertanra van szükség, mint egy irodalmi vagy jogász pályán. Az előbbi előnye, hogy megfelelő háttér biztosításával nagyban javítható a tanulási görbe, azonban sajnos az egyetemi körülmények között nincs lehetőség minden halltóval személyesen foglalkozni, hiszen ez óriási többlet munkát róna az oktatókra. Emiatt a hallgatók eredményes, mégis időtakarékos támogatása érdekében egyre nagyobb törekvés indult az automatizálása felé. \cite{NEEDED} Ezen megoldások számos előnnyel rendelkeznek:

\begin{itemize}
    \item Az értékeléshez nincs szükség a beadások letöltésére, azok saját számítógépen történő fordítására, futtatására, majd kiértékelésére.
    \item Azonnali visszajelzés a beadás sikerességéről a hallgatóknak.
    \item Egyéb beadáshoz kapcsolható metrikákkal is dolgohatunk: futási idő, memóriahasználat, stb.
    \item Automatikus pontszámítás a számított metrikák alapján.
    \item Határidők automatikus kezelése.
    \item Kommunikációs felület a hallagtó és oktató között.
\end{itemize}

\section*{Más automatikus kiértékelő rendszerek}\addcontentsline{toc}{section}{Más automatikus kiértékelő rendszerek}

\subsection*{WebAssign rendszer}\addcontentsline{toc}{subsection}{WebAssign rendszer}
A WebAssign \cite{WebAssign} rendszer a Hagen-i FernUiversität-en 1996 óta folyamatos fejlesztés alatt áll.

\subsection*{Moodle}\addcontentsline{toc}{subsection}{Moodle}
A Moodle\cite{Moodle} egy teljes körű eLearning rendszer, mely nyílt forráskódú GNU GPL \cite{GNUGPL} licenc alatt készült. A Moodle-nek mára több, mint 120 ezer felhasználója van, a világ 232 országában \cite{MoodleStats}, ami a nyílt forráskód előnyeivel kombinálva óriási potenciált jelent. 

A Moodle általános lehetőséget biztosít az online oktatáshoz. Egyszerűen hozhatunk létre benne kurzusokat, melyekre a jelentkezést akár korlátozhatjuk is. A kurzusokon belül témakörökre bonthatjuk a tananyagot, melynek formája sokféle lehet: pdf fájlok, videók, külső weboldalak, stb. A résztvevők elsajátított tudásának ellenőrzése sem marad el: a Moodle általános rendszert biztosít online tesztek készítésére is, különféle jelleggel, mint pl. több válaszlehetőségből helyes(ek) kiválasztása, egy soros vagy éppen hosszabb saját szavas válaszok. A legtöbb fajta tesztnél lehetőségünk van a helyes válaszok megadására, így a portál azonnal ki is értékeli a beadást, ennek köszönhetően pedig a felhasználó azonnal értesül az elért eredményéről.

Elterjedtségének köszönhetően számos közösségi fejlesztésű modul is készült hozzá, melyek közül találhatunk szép számmal a programozás oktatására fókuszolóakat is. Ilyen például a Virtual Programming Lab \cite{VPL} \cite{VPLJournal}, mely támogat forráskódszerkesztést a böngészőben, programok futtatását, ellenőrzését és plágium ellenőrzést is. Mindezt a felhasználók a már megszokott Moodle környezetben érhetik el a kényelmes használat érdekében.

\subsection*{EdX, Open Edx}\addcontentsline{toc}{subsection}{EdX, Open edx}
Az EdX \cite{EDX} egy nonprofit online kezdeményezés, melynek segítségével egyetemi szintű kurzusokat tartanak világszerte. Alapító tagjai a Harvard Univer

\section*{A szakdolgozat felépítése}\addcontentsline{toc}{section}{A szakdolgozat felépítése}
%----------------------------------------------------------------------------
\chapter{A Jporta (5-10 oldal)}\label{chapter:jporta}
%----------------------------------------------------------------------------

Mi is a jporta, milyen funkciói vannak, gui...
%----------------------------------------------------------------------------
\chapter{Hallgatói értékelési rendszer}\label{chapter:assessments}
%----------------------------------------------------------------------------

A korábban leírtaknak megfelelően a JPorta tárgyaiban lehetőség van különböző értékeléseket felvenni, majd azokat kurzusokhoz rendelni. Ezeket hozzárendelés után a kurzus oktatói tudják értékelni (\ref{fig:jporta_course_results}. ábra).

Új értékelést csak a tárgy adminisztrátorai tudnak létrehozni és kurzusokhoz rendelni. Az ehhez tartozó felület \aref{fig:jporta_add_result}. ábrán látható. Minden értékeléshez az alábbi tulajdonságok tartoznak:
\begin{itemize}
    \item Név: rövid név, mely azonosítja az értékelést, pl. ZH 1
    \item Típus: előre definiált értékek, melyekhez tartozik egy reguláris kifejezés \cite{RegExp}. Csak olyan értéket vehet fel, ami illeszkedik a hozzá tartozó kifejezésre.
    \item Súly: meghatározza a sorrendet az értékelések megjelenítésénél.
    \item Ki értékelheti: kurzusokhoz, vagy csak a tárgyhoz rendelt oktatók értékelhetik.
    \item Dinamikus-e: az adott értékelés dinamikusan értékelődik-e ki, ld. \aref{section:dynamic-assessments} pontban.
    \item Privát-e: a privát értékeléseket csak az oktatók látják, a hallgatók nem.
    \item Megjegyzés: részletes leírása az értékelésnek, tipikusan dinamikus értékelések esetén hasznos.
    \item Kurzusok: tárgyon belül mely kurzusokhoz akarjuk hozzárendelni az értékelést.
\end{itemize}

\begin{figure}[h]
    \centering
    \resizebox{\textwidth}{!}{
        \includegraphics[]{jporta_course_results.png}
    }
    \caption{Hallgatók értékelései}
    \label{fig:jporta_course_results}
\end{figure}

Ezen tulajdonságoknak köszönhetően az értékeléseket egyszerűen és személyreszabhatóan lehet kezelni. Alapvető céljuk a zárthelyi számonkérések eredményének adminisztrálása, de bármikor hozzáadhatunk egyéb mezőket is. Ilyen lehet pl. a hallgató házi feladatának személyes bemutatására kijelölt időpont, vagy éppen a házi feladat dokumentáció státusza.

Az univerzalitás egyedüli határa a megfelelő típus megtalálása az értékeléshez, de ez könnyen bővíthető. Új igény felmerülésekor a portál adminisztrátorai hozzáadhatnak új típust, mely egy tetszés szerinti reguláris kifejezésre illeszkedő tartalmat vár.

\begin{figure}[p]
    \centering
    \resizebox{0.9\textwidth}{!}{
        \includegraphics[]{jporta_add_result.png}
    }
    \caption{Értékelés típus hozzáadása}
    \label{fig:jporta_add_result}
\end{figure}

\section{Dinamikus értékelések}\label{section:dynamic-assessments}

Az értékelések létrehozásánál hamar felmerült az igény a dinamikus, azaz automatikusan számolódó mezők használtára. Ezek nagyban megkönnyítik a félév végi összesítést és a végső jegy meghatározását. Viszont ahhoz, hogy ezt széleskörűen lehessen használni egy teljesen általános rendszert kellett fejleszteni, hiszen minden tárgynak eltérő, akár félévről félévre változó követelményei lehetnek, melyekhez más és más számolásokat, súlyozásokat kellhet végezni. 

Ennek megoldására a jelengi implementáció előtt is volt mód, ám az fapadosnak számított. Először az oktatóknak exportálni kellett a meglévő eredményeket egy Excel fájlba, majd ezt a fájlt külső programmal szerkesztve kellett létrehozniuk a dinamikus mezők értékeit. Csak ezután tölthették fel a portálra a bővített fájlt, melynek feldolgozása során a JPorta az ebben található értékeket frissítette. Ezt a módszert Kálmán Viktor cserélte le a jelenlegi alternatívára \cite{KalmanMsc}.

\section{Dinamikus értékelések jelenlegi implementációja}
Az aktuális megvalósításnak két fő követelményt támasztottak: webes felületen beállíthatónak és az oktatók számára könnyen testreszabhatónak kellett lennie.

Ezek teljesítése érdekében a dinamikus mezők kiszámolására Python nyelven írt kódokat használ a portál. Minden dinamikus mezőhöz webes felületen módosítható kód tartozik, amely megkapja a hallgatókhoz tartozó adatokat, majd ez alapján kiszámolja a mező értékét. A megadott Python kódnak tartalmaznia kell egy \textit{calculate\_result} függvényt, mely három szótár (dictionary) típusú paraméterrel rendelkezik:

\begin{enumerate}
    \item paraméter a jelenléteket tartalmazza külön előadás, laboratóriumi és gyakorlati órákra bontva. Minden típushoz megtalálható, hogy hány alkalommal volt jelen a hallgató (attended), hány alkalommal lett megtagadva a jelenléte (denied), hány alkalommal nem jelent meg (no), illetve hogy összesen hány foglalkozás volt tartva abból a típusból (all).
    \item paraméter a (nem dinamikus) értékeléseket taralmazza, \textit{None} értéket ott, ahol nincs megadva eredmény. A kulcs mindig az adott értékelés azonosítója.
    \item paraméterben pedig az automatikusan kiértékelődő feladatok eredményei találhatóak. A kulcs itt is mindig az adott feladat azonosítója, értékei pedig az alábbiak: 
    \begin{itemize}
        \item \textit{None}, ha az adott feladatra nem adott be megoldást a hallgató.
        \item \textit{0}, ha az adott feladat kiértékelés alatt van.
        \item \textit{1}, ha az adott feladat oktatói jóváhagyásra vár.
        \item \textit{2}, ha az adott feladat sikertelen.
        \item \textit{3}, ha az adott feladat sikeresen lefutott.
    \end{itemize}
\end{enumerate}

Ez a függvény minden hallgatóra egyesével fog meghívódni, visszatérési értéke pedig megadja a mező értékét az adott hallgatónál.

Az alábbi példa kód a Programozás alapjai 3. tárgynál használt egyik dinamikus mezőhöz tartozik, feladata a félév végi átlag kiszámítása. Itt a félév során megírt 6 db zárthelyi dolgozatból a legjobb 4 átlagát kell venni, majd kerekíteni 2 tizedesjegyre. A további kerekítést a laborvezető oktatók végzik.

\begin{lstlisting}
def calculate_result(atds, asms, asgs):
    results = [asms[374], asms[375], asms[376], asms[377], asms[378], asms[379]]
    results = [x for x in results if x is not None]
    results.sort(reverse=True)
    return round(sum(results[:4])/4, 2)
\end{lstlisting}

Természetesen a kötelező \textit{calculate\_result} függvény mellett tetszőleges számban használhatunk segédfüggvényeket is a számolás átláthatósága és egyszerűsítése érdekében.

\section{Dinamikus értékelések tesztelése}
Az így elkészített dinamikus mezőben könnyen előfordulhatnak programozói hibák, emiatt a kód felhasználása előtt elengedhetetlen annak átfogó tesztelése. Szerencsére a korábban elkészült rendszer ezt is támogatja. \Aref{fig:jporta_dynamic_test}. ábrán látható módon meg tuduk adni teszt bemeneti paramétereket, majd a rendszer (az eredeti kiértékeléssel megegyező módon) kiszámolja a dinamikus mezőhöz tartozó értéket, így meggyőződhetünk a helyes működéséről.

\begin{figure}[h]
    \centering
    \resizebox{\textwidth}{!}{
        \includegraphics[]{jporta_dynamic_test.png}
    }
    \caption{Dinamikus mező tesztelése}
    \label{fig:jporta_dynamic_test}
\end{figure}

Az értékek mellett láthatjuk a programunk kimeneteit és visszatérési értékét is. Ezeket akár a teszteléshez is felhasználhatjuk, azonban élesítés előtt célszerű ezeket törölni, mivel a portál minden éles futásról tárolja az ilyen jellegű adatokat az ellenőrizhetőség miatt.

Ugyan a példában nem volt szükség sem a jelenlétekre, se az automatikusan kiértékelődő feladatok eredményére, ezeket a számonkérésekhez hasonlóan fel tudjuk használni, illetve a tesztelés is az előzőekkel megegyezően működik.

\section{Dinamikus mezők dinamikus függőségei}
Ez a rendszer az alapvető igényeket ugyan kielégíti, de hamar felmerült két fő hiányossága:

\begin{itemize}
    \item Dinamikus mezők kiszámítására nem használhatunk más dinamikus mezőket. 
    \item Jelenleg nem lehet egy hallgatóra vagy kurzusra újraszámolni a dinamikus mezők értékeit, hanem mindig a tárgy összes hallgatójára újra kiértékelésre kerülnek ezek. Kezdetben ez nem tűnt nagy pazarlásnak, mivel 10-20 másodpercnél többet nem vesz igénybe a művelet egy 400 fős tárgy esetében \cite{KalmanMsc}, azonban felhasználva az előző pont miatt szükséges módosításokat, már sokkal közelebb kerülünk ennek a pontnak a megoldására is.
\end{itemize}

Az első hiányosság miatt, ha egy tárgy esetében szeretnénk, hogy a hallgatók láthassák külön a zártyhelyi dolgozataik átlagát, majd ezen érték és más mezők alapján egy másik mezőt is hozzá szeretnénk adni, akkor ezt csak kódduplikációval tehetjük meg. Azaz mindkét mezőnél meg kell adnunk ugyan azt a Python kód részletet, illetve a második mezőnél ezt még ki kell egészítenünk. Ez számos problémát felvet: a kód nem újrahasználható, módosítások hibalehetőséget rejtenek, illetve akár inkonzisztens értékeket is előállítatunk, ha a kódokat nem sikerül szinkronban tartani.

A szakdolgozatom keretében erre a problémára is megterveztem és implementáltam egy elvárásoknak megfelelő megoldást. 

%----------------------------------------------------------------------------
\chapter{Automatizált feladatkiértékelő modul}\label{chapter:exercise}
%----------------------------------------------------------------------------

\section{Megvalósult funkciók}
\subsection{Jogosultságok felülvizsgálata (2-3 oldal)}
Acl, django permissions, stb...

\subsection{Kód lefedettség ellenőrzés (2-3 oldal)}

\subsection{Feladatok (tesztek) importálása és exportálása (2-3 oldal)}

\subsection{Feladatok csoportosítása (2 oldal)}

\section{További lehetőségek}

\subsection{Plágiumkeresés (1-2 oldal)}

\subsection{Verziókezelő támogatás (1-2 oldal)}
%----------------------------------------------------------------------------
\chapter*{Összefoglalás}\addcontentsline{toc}{chapter}{Összefoglalás}
%----------------------------------------------------------------------------

A szakdolgozatom keretében részletesen megismertem mind a JPorta rendszerét, mind a felé támasztott elvárásokat oktatói és hallgatói szempontból. Ezek alapján főként az aktuálisan legkritikusabbnak ítélt területek fejlesztésével foglalkoztam, azaz a dinamikus mezőkkel és az automatikusan kiértékelődő feladatok rendszerével.

Előbbin elért eredményeknek köszönhetően az oktatók már létrehozhatnak olyan dinamikus értékelési mezőket, melyek más dinamikus mezőket is használnak a számításaik során. Ezzel segítik az oktatókat a bonyolultabb mezők elkészítésében. Tipikusan ez a félév végi eredmény meghatározásánál lehet különösen hasznos.

Az automatikus kiértékelő rendszer fejlesztésének köszönhetően már lehetőség van oktatóknak és hallgatóknak is jogosultságot adni a feladatok létrehozására. Ez egy további lépést jelent a tárgyak széleskörű támogatására, hiszen így a feladatok elkészítésére nem csak a portál adminisztrátorait lehet megkérni, hanem minden arra jogosult személy kísérletezhet.

A kódlefedettség ellenőrző pedig lehetőséget teremt a C és C++ nyelvű feladatok esetén a tesztesetek értékelésére. Így könnyebb automatikusan kiszűrni a hibákat, működési zavarokat a beadott megoldásokban.

Végezetül pedig további hasznos funkciókra tettem javaslatot, melyek tovább növelhetik az oktatók és hallgatók elégedettségét a portállal szemben.
%----------------------------------------------------------------------------
\chapter*{Köszönetnyilvánítás}\addcontentsline{toc}{chapter}{Köszönetnyilvánítás}
%----------------------------------------------------------------------------

Szerenték köszönetet mondani konzulensemnek, Dr.~Szeberényi Imrének, aki a tanulmányaim és szakdolgozatom készítésének során végig támogatott.

Emellett külön köszönet jár a BME Informatikai Központ munkatársainak és volt munkatársainak, köztük Dudás Ádámnak és Kálmán Viktornak, akik a korábbi években elkészítették a JPorta fő moduljait és ezekkel kapcsolatban a félév során hasznos információkkal segítették munkámat.
\listoffigures\addcontentsline{toc}{chapter}{Ábrák jegyzéke}
%\listoftables\addcontentsline{toc}{chapter}{Táblázatok jegyzéke}
%\printglossary[title=Rövidítések jegyzéke]

\bibliography{bibliography}\addcontentsline{toc}{chapter}{Irodalomjegyzék}
\bibliographystyle{huplain}

\label{page:last}
\end{document}
